WPI's Marketing \& Communications office offers clear guidelines for designs that represent the institution's standards.
The guidelines highlight the power of consistency for all WPI representations and graphics.
Our final product aims to comply with the guidelines to the furthest extent possible.
From the Marketing \& Communications office:
% TODO: Cite.
\begin{displayquote}
    Marketing Communications offers a number of tools, templates, and self-serve resources to help the campus community create unified, branded, and targeted communications for your audiences.
\end{displayquote}
Read more about the guidelines and find more resources \href{https://www.wpi.edu/offices/marketing-communications/resources}{here}.

\SubSubSection{Color}{methodology}{conceptual_designs}{wpi_guidelines}{color}

    % TODO: Cite.
    % TODO: PMS glossary.
    % TODO: Pantone registered symbol.
    WPI's official colors use the Pantone Matching System (PMS).
    The colors are Crimson (PMS \texttt{187c}) and Gray (PMS \texttt{429c}).
    Black is used as an accent color.
    \textit{Note that colors in this document may be inaccurate due to colorspaces.}

    % TODO: Cite.
    \begin{Figure}{Table}{H}
        \centering
        \begin{tblr}{
            hlines,
            vlines
        }
            \textbf{Color}               & \textbf{Identifier}     & \textbf{Pantone} & \textbf{CMYK}        & \textbf{RGB}         & \textbf{HTML} \\
            \ColorBox{Primary}{1cm}{1cm} & Primary                 & \texttt{187c}    & \texttt{7,100,82,26} & \texttt{172,43,55}   & \texttt{ac2b37} \\
            \ColorBox{Accent}{1cm}{1cm}  & Accent                  &                  & \texttt{0,0,0,100}   & \texttt{0,0,0}       & \texttt{000000} \\
            \ColorBox{Gray}{1cm}{1cm}    & Gray                    & \texttt{429c}    & \texttt{21,11,9,23}  & \texttt{169,176,183} & \texttt{a9b0b7} \\
            \ColorBox{Body}{1cm}{1cm}    & Body                    &                  & \texttt{12,5,0,7}    & \texttt{52,56,59}    & \texttt{34383b} \\
        \end{tblr}
        \caption{WPI Colors}
        \label{fig:wpi_colors}
    \end{Figure}

\SubSubSection{Logos}{methodology}{conceptual_designs}{wpi_guidelines}{logos}

    % TODO.
    WPI LOGOS.

\SubSubSection{Fonts}{methodology}{conceptual_designs}{wpi_guidelines}{fonts}

    WPI guidelines specify two fonts: \href{https://fonts.adobe.com/fonts/minion}{Minion Pro} and \href{https://fonts.adobe.com/fonts/myriad}{Myriad Pro}.
    Minion Pro is used for formal, authoritative, and traditional text.
    Myriad Pro is used for informal, youthful, and friendly text.

    \begin{Figure}{Table}{H}
        \centering
        \begin{tblr}{
            hlines,
            vlines
        }
            \textbf{Font Family}     & Minion Pro                                      & Myriad Pro \\
            \textbf{Regular}         & {\MinionPro Minion Pro}                         & {\MyriadPro Myriad Pro} \\
            \textbf{Italic}          & {\MinionPro\itshape Minion Pro}                 & {\MyriadPro\itshape Myriad Pro} \\
            \textbf{Medium}          & {\MinionPro\FontVariant{md}Minion Pro} \\
            \textbf{Medium Italic}   & {\MinionPro\FontVariant{md}\itshape Minion Pro} \\
            \textbf{Semibold}        & {\MinionPro\FontVariant{sb}Minion Pro}          & {\MyriadPro\FontVariant{sb}Myriad Pro} \\
            \textbf{Semibold Italic} & {\MinionPro\FontVariant{sb}\itshape Minion Pro} & {\MyriadPro\FontVariant{sb}\itshape Myriad Pro} \\
            \textbf{Bold}            & {\MinionPro\FontVariant{bx}Minion Pro}          & {\MyriadPro\FontVariant{bx}Myriad Pro} \\
            \textbf{Bold Italic}     & {\MinionPro\FontVariant{bx}\itshape Minion Pro} & {\MyriadPro\FontVariant{bx}\itshape Myriad Pro} \\
        \end{tblr}
        \caption{WPI Font Families}
        \label{fig:wpi_font_families}
    \end{Figure}

    These fonts are utilized for different parts of brand content.

    % TODO: Cite.
    \begin{Figure}{Table}{H}
        \centering
        \begin{tblr}{
            hlines,
            vlines
        }
            % TODO: Fonts.
            \textbf{Identifier} & \textbf{Font}       & \textbf{Example} \\
            Formal Primary      & Minion Pro Semibold & {\MinionPro\FontVariant{sb}\color{Primary}ABCDEFGHIJKLMNOPQRSTUVWXYZ\\abcdefghijklmnopqrstuvwxyz\\1234567890} \\
            Formal Secondary    & Minion Pro Italic   & {\MinionPro\itshape\color{Primary}ABCDEFGHIJKLMNOPQRSTUVWXYZ\\abcdefghijklmnopqrstuvwxyz\\1234567890} \\
            Informal Secondary  & Myriad Pro Semibold & {\MyriadPro\FontVariant{sb}\color{Primary}ABCDEFGHIJKLMNOPQRSTUVWXYZ\\abcdefghijklmnopqrstuvwxyz\\1234567890} \\
            \SetCell[c=3]{c}\textbf{\itshape Specific Elements} \\
            Headline            & Minion Pro Semibold & {\MinionPro\FontVariant{sb}\color{Primary}Headline} \\
            Informal Headline   & Myriad Pro Semibold & {\MyriadPro\FontVariant{sb}\color{Primary}Informal Headline} \\
            Sub-headline        & Minion Pro Italic   & {\MinionPro\itshape\color{Primary}Sub-headline} \\
            Body                & Myriad Pro Regular  & {\MyriadPro\color{Body}Body}
        \end{tblr}
        \caption{WPI Font Usages}
        \label{fig:wpi_font_usages}
    \end{Figure}