%Introduction
The George C. Gordon Library is a vital part of the college experience at Worcester Polytechnic Institute. The library staff work tirelessly to maintain and reinforce quality educational services, as well as strive to help students directly with boundless academic concerns. The staff utilize various web-based product suites to host and organize the various services they offer. However, many library staff members view some of these products as counterproductive and challenging to operate efficiently within.

Simply put, the library staff use these tools daily and have found numerous issues that are of high demand to be solved. For example, LibraryH3lp, a popular tool among libraries, is often reported to have bugs and that it is rather difficult to use. Additionally, the library staff utilizes a handful of the products from SpringShare. Nonetheless, these tools often prove to perform unreliably and cause more problems than they solve.

Despite the challenges posed by these tools, the library staff continue to work hard to provide the best possible service to WPI students. They are constantly trying to find new and improved ways to work around the challenges presented by these tools. Therefore, we sought to step in with this project to ease the troubles that the WPI library staff encounter.

Both participants of this MQP together completed a closely related IQP. The IQP's goal was to develop a mobile application for the Gordon Library that collects the library's most commonly used services into one convenient place. With the extensive research and feedback the Gordon Library's staff provided us during our IQP, we began our MQP with ample information about the systems that worked and those that did not.

The LibraryH3lp platform was reportedly riddled with pitfalls pertaining to the user's experience on both the administrative and client sides which needed more functionality for what the Gordon Library required to best provide students with quality educational services. A change was in order to match the standard of what the library continually aims to maintain with the tools they used. This project set out to implement a solution to such problems, thus developing a web application that would replace LibraryH3lp and its current library chat application. We used the existing application as a foundation for reference of features and capabilities to then build an overhauled version that included a number of improvements tailored to what the library staff had detailed.

We organized a number of meetings with the library staff during this project. Our initial meeting during the beginning of the project was used to establish a solid baseline to best structure the issues with the current system. From there, could build a skeleton of the system we were planning to develop along with storyboards in an adolescent state. Then, we held a second meeting with the library staff once these storyboards were completely fleshed out. During this meeting we discussed and reinforced the concepts that were brainstormed at the beginning of the project and demonstrate them though storyboards. The feedback on them was positive and we received additional options we may have, though just out of the scope of our jurisdiction, but would enhance our application even more.